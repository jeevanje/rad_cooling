%% Draft settings
\documentclass[10pt]{article}
\usepackage{amsmath}
\usepackage{amssymb}
\usepackage{graphicx}
\usepackage{subfigure}
\usepackage{color}
 \usepackage{lineno}
\usepackage{simplemargins}
\usepackage{natbib}

% \linenumbers*[1]
 \usepackage[T1]{fontenc} % For citing barki{\dj}ija
\setkeys{Gin}{draft=false}


% Margins
\setleftmargin{1in}
\setrightmargin{1in}
\setbottommargin{1in}
\settopmargin{1in}
 
% My Commands
%\input{tensor_book_shrtcts.tex}
\newcommand{\comment}[1]{\textcolor{blue}{[{#1}]}}


%%Nadir's Shortcuts
\newcommand{\beqn}{\begin{equation}}
\newcommand{\eeqn}{\end{equation}}
\newcommand{\beqa}{\begin{eqnarray}}
\newcommand{\eeqa}{\end{eqnarray}}
\newcommand{\beqanonum}{\begin{eqnarray*}}
\newcommand{\eeqanonum}{\end{eqnarray*}}
\newcommand{\beqnonum}{\begin{equation*}}
\newcommand{\eeqnonum}{\end{equation*}}
\newcommand{\jump}{\vspace{0.5cm}}
\newcommand{\bbf}{\begin{bf}}
\newcommand{\ebf}{\end{bf}}
\newcommand{\eqnref}[1]{(\ref{#1})}
\newcommand{\defn}[1]{\begin{bf}\emph{#1}\end{bf}}
\newcommand{\reals}{\ensuremath{\mathbb{R}}}
\newcommand{\complex}{\ensuremath{\mathbb{C}}}
\newcommand{\integers}{\ensuremath{\mathbb{Z}}}
\newcommand{\half}{\ensuremath{\frac{1}{2}}}
\newcommand{\n}{\nonumber}
\newcommand{\inverse}{^{-1}}

%calculus shorthand
\newcommand{\timeder}{\frac{d}{dt}}
\newcommand{\partialder}[1]{\frac{\partial}{\partial #1}}
\newcommand{\partialderf}[2]{\ensuremath{\frac{\partial #1}{\partial #2}}}
\newcommand{\der}[2]{\ensuremath{\frac{d #1}{d #2}}}
\newcommand{\dx}{\ensuremath{\frac{d}{dx}}}
\newcommand{\ddx}{\ensuremath{\frac{d}{dx}}}
\newcommand{\kvec}{\ensuremath{\vec{k}}}
\newcommand{\uvec}{\ensuremath{\mathbf{u}}}
\newcommand{\zhat}{\ensuremath{\mathbf{\hat{z}}}}
\newcommand{\khat}{\ensuremath{\mathbf{\hat{k}}}}
\newcommand{\unitvect}[1]{\ensuremath{\mathbf{\hat{#1}}}}
\newcommand{\ppx}{\ensuremath{\partial_x}}
\newcommand{\ppy}{\ensuremath{\partial_y}}
\newcommand{\ppz}{\ensuremath{\partial_z}}
\newcommand{\ppt}{\ensuremath{\partial_T}}
\newcommand{\ppp}{\ensuremath{\partial_p}}


% radiation shorthand
\newcommand{\cotwo}{\ensuremath{\mathrm{CO_2}}}
\newcommand{\htwo}{\ensuremath{\mathrm{H_2O}}}
\newcommand{\QLW}{\ensuremath{Q_\mathrm{LW}}}
\newcommand{\QSW}{\ensuremath{Q_\mathrm{SW}}}
\newcommand{\Qnet}{\ensuremath{Q_\mathrm{net}}}
\newcommand{\FLW}{\ensuremath{F^\mathrm{LW}}}
\newcommand{\FSW}{\ensuremath{F^\mathrm{SW}}}
\newcommand{\USW}{\ensuremath{U^\mathrm{SW}}}
\newcommand{\DSW}{\ensuremath{D^\mathrm{SW}}}
\newcommand{\Fnet}{\ensuremath{F^\mathrm{net}}}
\newcommand{\olr}{\ensuremath{\mathrm{OLR}}}
\newcommand{\OLR}{\ensuremath{\mathrm{OLR}}}
\newcommand{\trans}{\ensuremath{\mathcal{T}}}
\newcommand{\solar}{\ensuremath{I_0}}
\newcommand{\cool}{\ensuremath{\mathcal{C}}}
\newcommand{\cminverse}{\ensuremath{\mathrm{cm^{-1}}}}
\newcommand{\pierre}{P10}
\newcommand{\tauk}{\ensuremath{\tau_k}}
\newcommand{\Wmsq}{\ensuremath{\mathrm{W/m^2}}}

% meteorology shorthand
\newcommand{\qv}{\ensuremath{q}}
\newcommand{\rhov}{\ensuremath{\rho_\mathrm{v}}}
\newcommand{\Hv}{\ensuremath{H_\mathrm{v}}}
\newcommand{\Rv}{\ensuremath{R_\mathrm{v}}}
\newcommand{\qa}{\ensuremath{q_a}}
\newcommand{\qvstar}{\ensuremath{q^*}}
\newcommand{\Ta}{\ensuremath{T_a}}
\newcommand{\Tav}{\ensuremath{T_\mathrm{av}}}
\newcommand{\Ts}{\ensuremath{T_\mathrm{s}}}
\newcommand{\ps}{\ensuremath{p_s}}
\newcommand{\RH}{\ensuremath{\mathrm{RH}}}
\newcommand{\WVP}{\ensuremath{\mathrm{WVP}}}
\newcommand{\ztop}{\ensuremath{z_\mathrm{top}}}
\newcommand{\ztp}{\ensuremath{z_\mathrm{tp}}}
\newcommand{\zlcl}{\ensuremath{z_\mathrm{LCL}}}
\newcommand{\Tlcl}{\ensuremath{T_\mathrm{LCL}}}
\newcommand{\Ttp}{\ensuremath{T_\mathrm{tp}}}
\newcommand{\ptp}{\ensuremath{p_\mathrm{tp}}}
\newcommand{\lapseav}{\ensuremath{\Gamma_\mathrm{av}}}
\newcommand{\gammaav}{\ensuremath{\Gamma_\mathrm{av}}}
\newcommand{\Kinverse}{\ensuremath{\mathrm{K^{-1}}}}
\newcommand{\Htauk}{\ensuremath{H_{\tau_k}}}

%Variables
\newcommand{\figurepath}{../figures/}


\begin{document}

%% ------------------------------------------------------------------------ %%
%
%  TITLE
%
%% ------------------------------------------------------------------------ %%


\title{Invariant Flux Divergences and a Simple Estimate of Precipitation Change with Surface Warming}

%% ------------------------------------------------------------------------ %%
%
%  AUTHORS AND AFFILIATIONS
%
%% ------------------------------------------------------------------------ %%


 \author{Nadir Jeevanjee\footnote{Department of Physics, University of California at Berkeley, Berkeley, CA 94702  USA. jeevanje@berkeley.edu (corresponding author)} \footnote{Climate and Ecosystems Science Division, Lawrence Berkeley National Laboratory, Berkeley, CA USA} and David Romps\footnote{Department of Earth and Planetary Sciences, University of California at Berkeley, Berkeley, CA 94702  USA.} \footnote{Climate and Ecosystems Science Division, Lawrence Berkeley National Laboratory, Berkeley, CA USA}
}

\maketitle

\begin{abstract}
We show that radiative flux divergences, when considered as functions of temperature as a vertical coordinate, are insensitive to surface temperature \Ts. This \Ts-invariance  leads to simple expressions for the \Ts-dependence of column-integrated radiative cooling and precipitation. In particular, these expressions allow us to predict and understand the roughly 3\% \Kinverse\ increase in mean precipitation with surface warming found in simulations of tropical convection, without requiring any radiative transfer calculation for the perturbed state. 

%\vspace{0.5cm}
%
%
\end{abstract}


%% ------------------------------------------------------------------------ %%
%
%  TEXT
%
%% ------------------------------------------------------------------------ %%


\section {Introduction}
Introduction here

\section{Simulations}
RCE simulations in a CRM at $\Ts=(280,290,300,310)$ K. Pre-industrial \cotwo\ of 280 ppm. No ozone.


%================%
% vapor density and    %
% optical depth            %
%================%
\section{Vapor density, vapor path, and optical depth}
	\subsection{Vapor density}
	The \Ts-invariance of various radiative quantities stems from the key fact that  the water vapor density 
	\beqn
		\rho_v =  \RH\, p_v^*(T)/R_ vT \; 
	\label{rhov}
	\eeqn
	 is, up to variations in relative humidity \RH, a function of temperature only. [Here $p_v^* = p_v^\infty \exp(-L/R_vT)$ is the vapor pressure of water ($p_v^\infty = 2.5\times 10^{11} $ Pa), and all other symbols have their usual meaning.] We verify this by plotting \rhov\ profiles from our RCE simulations, with both linear and log scales and using temperature as the vertical coordinate,  in Fig. \ref{rhov_fig}. Indeed the \rhov\ profiles at different \Ts\ collapse onto a single curve, i.e. the function $\rhov(T)$ has virtually no \Ts\ dependence.
	 
	The significance of this is that longwave radiative transfer is dominated by absorption and emission by water vapor, and the optical depth \tauk\ at any given wavenumber $k$ outside the \cotwo\ band is determined to a large extent by the \rhov\ profile. As we will demonstrate momentarily, the \Ts-invariance of $\rhov(T)$ implies a rough \Ts-invariance of $\tauk(T)$. Since radiative flux divergences are determined largely by the relationship between optical depth profiles and temperature profiles, this will yield a rough \Ts-invariance of radiative flux divergence profiles, when the latter are considered as functions of temperature.	 

	\subsection{Optical depth and vapor path}
	To understand and demonstrate the \Ts-invariance of \tauk\ and various flux divergences,  it is useful to develop an analytical expression for \tauk\ in terms of the water vapor path \WVP. We begin with a  formula for $\tau_k$ at height $z$,  given by  Eqn. (4.67) of \cite{pierrehumbert2010}, hereafter \pierre,
	\beqn
		\tau_k = \int_z^\infty \kappa(k) \frac{p(z')}{p_0} \rho_v\, dz'  
		\label{tau1}
	\eeqn
where $\kappa(k)$ is the mass absorption coefficient ($\mathrm{m^2/kg}$) at wavenumber $k$ , $\rho_v$ is the density of water vapor, and $p/p_0$  represent the effects of pressure scaling. (We neglect temperature scaling of absorption coefficients as it is entirely insensitive to \Ts; this omission does not affect our conclusions.)  The reference pressure  $p_0$ is set to 100 hPa.


To make the expression   \eqnref{tau1}  analytically tractable, we make several simplifications.  First, since the integral in \eqnref{tau1} is dominated by values of the integrand close to $z$, we evaluate $p$  at $z$ and pull it out of the integral. This allows us to write $\tau_k$ in terms of the water vapor path $\WVP\equiv \int_z^\infty  \rho_v\, dz'$ as
	\beqn
		\tau_k =  \kappa(k) \frac{p}{p_0} \cdot \WVP \; .
	\label{tauWVP}
	\eeqn
To estimate the water vapor path, we also view \WVP\ as a function of temperature only, and using \eqnref{rhov} write it as 
	\beqn
		\WVP(T) =\frac{ \RH\, p_v^\infty}{R_v \gammaav}\int^T_{\Ttp} \frac{e^{-L/R_v T'}}{T'}\; dT' \; 
	\label{WVP1}
	\eeqn	
where we assume a constant, average lapse rate $\gammaav$, and we take the lower limit of our integral to be the tropopause temperature \Ttp\ (this neglect of stratospheric water vapor does not affect our results). 

We make an approximate evaluation of the integral in \eqnref{WVP1} in the Appendix, obtaining
	\beqn
		\WVP(T) =\WVP_0\exp\left(-\frac{L}{R_vT}\right),\quad \mbox{where}\quad  \WVP_0 \equiv \frac{\RH\, p_v^\infty\, \Tav}{L\,\gammaav} \; .
	\label{WVP2}
	\eeqn
(The quantity $\WVP_0$ should be interpreted as a characteristic, though not necessarily realistic, value of \WVP, analogous to $p_v^\infty$ for $p_v^*$.) This approximate formula is discussed further in the Appendix, and  validated in Fig. \ref{wvp_check}. We can now  plug \eqnref{WVP2} into \eqnref{tauWVP}, but  since \WVP\ is naturally a function of temperature, we should express $\tau_k$ as a function of temperature as well. This means writing $p$ as 
	\beqn
		p=\ps(T/\Ts)^{g/R_d\gammaav} \; ,
	\label{pressure}
	\eeqn 
	which is the expression for $p(T)$ for an atmosphere with constant lapse rate $\gammaav$. We note here that we diagnose \gammaav\ for a given simulation by evaluating \eqnref{pressure} at \Ttp\ and solving for  \gammaav, i.e.
	\beqn
		\gammaav = \frac{g}{R_d}\frac{\ln (\Ttp/\Ts)}{\ln(\ptp/\ps)}
	\label{gamma_diag}
	\eeqn
	where \ptp\ is the pressure at $T=\Ttp$. Putting all this together finally yields our  analytical expression for $\tau_k$ in terms of elementary functions,
	\beqn
		\tau_k(T) = \kappa(k)\,  \frac{\ps}{p_0}\left(\frac{T}{\Ts}\right)^{\frac{g}{R_d\gammaav}}\cdot \WVP_0\exp\left(-\frac{L}{R_v T}\right) \quad .
	\label{tau2}
	\eeqn
\comment{add underbraces, units here} Though this expression is approximate, it contains no unobservable, tunable parameters. We check our approximations by comparing profiles of $\tau_k$ for  $k=(100,\ 500,\ 900)\ \cminverse$ as computed from both \eqnref{tau1} and \eqnref{tau2}; these are shown in solid and dashed lines respectively in the left panel of Figure \ref{tauk}. Though the biases in \WVP\ from Figure \ref{wvp_check} are certainly evident, the match between our analytical expression \eqnref{tau2} and the numerical calculation using \eqnref{tau1} is reasonable.

%The analytical expression  \eqnref{tau2} allows us to analyze the \Ts-dependence of \tauk. First, the pressure broadening factor, proportional to $p(T)$ as given by  \eqnref{pressure}, decreases as \Ts\ increases, both because of the direct dependence on \Ts\ and because \gammaav\ decreases with \Ts, at a rate of roughly $\der{\gammaav}{\Ts}= 1\times 10^{-4} \ \mathrm{m\inverse}$. Thus pressure broadening causes $\tauk(T)$ to decrease with increasing \Ts, at a rate of roughly 1.5 --  3\% \Kinverse\ (the higher rates apply in the upper troposphere where the lapse rate effect is largest). However, there is a countervailing increase of $\tauk(T)$ with \Ts\ due to the fact that $\WVP_0$ in \eqnref{tau2} depends on \gammaav, as given in \eqnref{WVP2}. This effect stems from the fixed tropopause temperature \Ttp, which means that as \gammaav\ decreases with warming, the vertical distance between a temperature $T$ and \Ttp\  (and hence the \WVP\ at $T$) are increasing. This effect brings the decrease of $\tauk(T)$ with \Ts\ down to 0 -- 2\% \Kinverse, with the largest changes still in the upper troposphere.

With the analytical expression \eqnref{tau2} for $\tauk(T)$ in hand we can now ask how this functions depends on \Ts. We might expect the vertical behavior of \tauk\ to be dominated by the exponential $\exp(-L/R_vT)$, which has no \Ts-dependence. The factor of
$(T/\Ts)^{(g/R_d\gammaav)}$ coming from pressure broadening does have some \Ts-dependence, both explicitly and implicitly through \gammaav\ (which decreases with increasing \Ts), but the exponent $g/R_d \gammaav \approx 1/5$ so the effect is mild. There is also a mild dependence of  $\WVP_0$ on \Ts\ through \gammaav. 

We thus expect a relatively small dependence of $\tauk(T)$ on \Ts. Figure \ref{tauk} confirms this, showing no discernible variation in $\tauk(T)$ for $T \gtrsim 260 $ K, and at most a factor of 2 variation in $\tauk(T)$ for $T \lesssim 260$ K over our set of RCE experiments. The latter may sound substantial, but note that by \eqnref{tau2} \tauk\ varies roughly exponentially at a rate of $L/R_vT^2 \approx  12\%\ \Kinverse$ in the upper troposphere ($T=220$ K), or a doubling every 6 K. This means that the temperature at which a given value of  \tauk\ occurs changes by at most 6 K over our range of experiments, while \Ts\ exhibits a much larger range of  30 K.


%=================================%
% Temperature invariance of flux divergence %
%=================================%
\section{Temperature Invariance of Flux Divergences}
Now that we have established the insensitivity of  \tauk\  to \Ts, we turn to the \Ts-invariance of radiative flux divergence. Though flux divergence is usually defined as a $z$-derivative of net upwelling flux, we will again find it useful to use temperature as a vertical coordinate, and so we consider the $T$-derivative of net flux. To provide a basis for reasoning about this quantity, we first develop an analytical approximation for it via the cooling-to-space (CTS) approximation. This approximation is discussed in \cite{thomas2002}, pgs. 423-427, and has been evaluated in, e.g., \cite{fels1975,rodgers1966}.

	We begin with the spectral longwave (LW) flux divergence at wavenumber $k$, $-\ppt \FLW_k$ (we introduce a minus sign to maintain a consistent sign with  $\ppz \FLW_k$). The CTS approximation says that this  can be approximated as
	\beqn
		-\ppt \FLW_k \approx - \pi B_k \frac{d \trans_k}{dT}
	\label{cts_spectral}
	\eeqn
where  $\trans_k\equiv \exp(-\tauk)$ is the transmission function,  representing the fraction of radiation emitted at a given height which travels unabsorbed out to space. Utilizing our expression \eqnref{tau2} for \tauk\ and integrating over wavenumbers,  we obtain the (spectrally integrated) LW flux divergence
	\beqn
		(-\ppt \FLW)(T)  = \left( \frac{g}{R_dT\gammaav} + \frac{L}{R_v T^2}\right) \int dk\, \pi B_k(T)\, \tauk(T) e^{-\tauk(T)}  \; .
	\label{cts1}
	\eeqn
The function $\tauk \exp(-\tauk)$ has a relatively sharp peak at $\tauk=1$, reproducing the familiar rule that emission to space at a given height occurs at frequencies with roughly unit optical depth. In Appendix A we take advantage of this fact to approximate the integrand as a delta function, and obtain
	\beqn
		(-\ppt \FLW)(T)  = \pi B_{k_1}(T) \left( \frac{g}{R_dT\gammaav} + \frac{L}{R_v T^2}\right) \frac{\Delta k}{e} \; .
		\label{cts2}
	\eeqn
\comment{add underbraces and units}.

 Before proceeding let us gain a heuristic understanding of the quantities in Eqn. \eqnref{cts2}. The wavenumber $k_1(T)$ at which the Planck function $B_k(T)$ is evaluated  is the wavenumber  at height $T$ that has unit optical depth, i.e. the wavenumber around which we expect emission to space to be concentrated. The expression in parentheses is an inverse  `scale temperature' for optical depth of roughly $\mathrm{(6 -13\ K)}\inverse$, and gives the gradient of emissivity in temperature coordinates. The factor of $1/e$ is just the value of $\tauk \exp(-\tauk)$ evaluated at $k=k_1$. Finally, the factor $\Delta k \approx 200\ \cminverse$ is the characteristic  width of the spectral region at any given height that is actively radiating to space.  This latter agrees nicely with the spectral LW cooling plots found in, e.g., \cite{huang2013,clough1992}.
 
 	A numerical analysis of the factor $B_{k_1}(T)$ in \eqnref{cts2}, utilizing an inversion of \eqnref{tau2} for $k_1(T)$, shows that it decreases by at most .3\% \Kinverse\ with \Ts. The scale temperature exhibits similarly small (but oppositely signed) variations with \Ts, caused by decreases in \gammaav. This suggests that variations of $(-\ppt \FLW)(T)$ with \Ts\ should be barely discernible, which we confirm in Fig.  \ref{pptflw_tinv_dam}. That figure also plots 
$-\ppt \FLW$ as functions of $z$ and $p$, to emphasize the simplification that occurs when $T$ is used as the vertical coordinate.

A similar argument holds for the shortwave (SW). If $I_k$ is the incident solar flux at wavenumber $k$, then without approximation we have
	\beqn
		-\ppt \FSW_k = - I_k \der{\trans_k}{T}
		\n
	\eeqn
\citep[c.f.][eqn. 9.26]{thomas2002}. This is similar to  \eqnref{cts_spectral} but with $B_k(T) \rightarrow I_k$. As a function of $k$, $I_k$ behaves very much like a translated (in $k$-space) version of  $B_k(T)$,  where $T$ is any tropospheric temperature  \citep[cf. Fig. 3.2 of][]{hartmann1994book}. Furthermore, the slopes of $\ln(\kappa(k))$ in the far-IR bands, shown in Fig. \ref{kappa_h2o}, are similar to the slopes in the near-IR bands (Fig. 5.13 of \pierre). These similarities imply that the insensitivity of $(-\ppt \FSW)(T)$ to \Ts\ should be comparable to that of  (-\ppt \FLW)(T).
 This is confirmed in Fig. \ref{pptfsw_tinv_dam}, where again the simple behavior of $(-\ppt \FSW)(T)$ is contrasted with that of $(-\ppt \FSW)(z)$ and $(-\ppt \FSW)(p)$.


%=================%
% Simple picture for Q   %
%=================%
		
\section{A simple picture for column-integrated radiative cooling}
	We now employ the \Ts-invariance of radiative flux divergences  to construct a simple, quantitative picture of how column-integrated radiative cooling, and hence precipitation,  change with surface temperature. 
	
	Let $F$ denote a radiative flux in a particular channel -- LW, SW, or net (LW+SW) -- and $Q$ the associated column-integrated cooling. The basic idea is to write $Q$ as an integral of $-\ppt F$ over the free troposphere (i.e. above the lifting condensation level \Tlcl; see motivation for this below) in temperature coordinates: 
	\beqn
		Q = - \int_{\Ttp}^{\Tlcl} \partial_{T'} F dT' \ . 
		\n
	\eeqn
  If we approximate the change in LCL temperature \Tlcl\ as equal to the change in \Ts, then the change in $Q$ with surface temperature is  simply
	\beqn
		\der{Q}{\Ts} \ =\  \left.  -\ppt F\right|_{\Tlcl}  \; .
	\label{dqdts}
	\eeqn
In other words, since the tropospheric cooling profile $(-\ppt F)(T)$  is independent of \Ts, increasing \Ts\ just exposes more of this profile.  The contribution of this new section of the $(-\ppt F)(T)$ curve to $Q$ is given by \eqnref{dqdts}.  For finite changes in \Ts\ this formula approximates $(-\ppt F)(T)$ in the newly exposed region as equal to $-\ppt F$ at the LCL of the base state, but for small enough changes in \Ts\ this approximation should be adequate.

Let us test the predictive power of \eqnref{dqdts} by checking if, given output from an RCE simulation at a particular \Ts, it can predict how $Q$ will change as $\Ts$ is increased. For definiteness we set $\partialderf{\gammaav}{\Ts}= 1\times 10^{-4} \ \mathrm{m\inverse}$, broadly consistent with the changes in our diagnosed values of \gammaav, and we  diagnose \Tlcl\ as $T$ at the low-level maximum in cloud fraction. With these diagnostics, the panels of Fig. \ref{Qnet_varsst} plot $Q(\Ts)$ as diagnosed directly from our CRM simulations, along with estimates of the slope of this curve derived using  Eqn. \eqnref{dqdts}, for the SW, LW, and net  channels. This figure shows that equation \eqnref{dqdts} does a good job of capturing the changes in  absorption in all channels. Thus our observation that radiative fluxes, when considered as functions of temperature, are roughly \Ts-invariant allows us to  predict how column-integrated cooling will change, via a simple analytic formula that requires only diagnostics from the base state. In particular, the perturbed state radiative transfer calculations employed in many previous studies \citep[e.g.][]{pendergrass2014,ogorman2012, held2006} are not required. 

Of course,  our interest in column-integrated radiative cooling stems from the atmospheric energy constraint on precipitation, which in its most precise form says that time and domain-mean precipitation $P$, when expressed in $\mathrm{W/m^2}$, should balance the net radiative cooling integrated over the free troposphere \citep{ogorman2012}, which is our \Qnet\ . We thus plot $P$ in the right panel of Fig. \ref{Qnet_varsst}, and see that $P$ tracks \Qnet\ as expected, and thus that Eqn. \eqnref{dqdts} can indeed provide a first-order, analytical estimate of how precipitation changes with warming, at least in our simplified RCE context.


%===========%
% Discussion    %		
%===========%

\section{Summary and discussion}
Our results can be summarized as follows:
\begin{itemize}
	\item The manifest \Ts-invariance of \rhov\ as a function of temperature, (cf. Eqn. \eqnref{rhov} and Fig. \ref{rhov_fig}), leads to a \Ts-invariance of optical depth, as seen in Fig. \ref{tauk}.
	\item This, combined with the cooling-to-space approximation, suggests that SW and LW flux divergences should also be \Ts-invariant functions of temperature, as confirmed in Figs. \ref{pptflw_tinv_dam} and \ref{pptfsw_tinv_dam}.
	\item This \Ts-invariance of flux divergences leads to a simple analytical formula, Eqn. \eqnref{dqdts}, for how column-integrated radiative cooling increases with \Ts. This formula is validated in Fig. \ref{Qnet_varsst},  and requires no perturbed state radiative transfer calculation.
	\end{itemize}

A particular motivation for this work was to try and understand the 1-3\% \Kinverse\ increase in global mean precipitation found in GCM warming experiments \citep{lambert2008,held2006}. Though we work here in an idealized RCE context, where values for this number are on the high end of that range \citep[e.g. the 3\% \Kinverse\ found by][]{romps2011}, we expect the same physics to be at work. 

Does our Eqn. \eqnref{dqdts} predict a 3\% \Kinverse\ increase in mean precipitation, consistent with \cite{romps2011} and the upper end of GCM projections? Specializing to \Ts=300 K, we have $\Qnet(300\ \mathrm{K})\ =\  104\ \Wmsq$, and Eqn. \eqnref{dqdts} gives $d\Qnet/d\Ts = 4.3 \ \Wmsq\Kinverse$, for a predicted increase in precipitation of 4\% \Kinverse. This is consistent with the value obtained by \cite{muller2011b}, but higher than our  3\% \Kinverse\  benchmark. The likely culprit here is the direct radiative forcing due to doubled \cotwo, present in the experiments of \cite{romps2011} but absent from our experiments and those of \cite{muller2011b}. This effect can be estimated by combining  a best-guess climate sensitivity of 3 K per doubling of \cotwo\ \citep{ipccspm2013} with a $2 \times \cotwo$ radiative forcing of $\sim$ 3.5 \Wmsq. Ignoring the weak radiative effect of \cotwo\ on the surface energy budget, we obtain a \cotwo\ direct effect on atmospheric cooling of  - 1 \Wmsq\ per degree of equilibrated \cotwo-induced warming, consistent with  \cite{pendergrass2014}. Taking this into account would bring our estimated precipitation change at $\Ts=300$ K down to 3\% \Kinverse, consistent with \cite{romps2011} and the upper end of GCM projections.

%===========%
% Appendix       %		
%===========%

\section*{Appendix}
\appendix
	\section{Analytic formula for water vapor path}		
	To evaluate the integral in \eqnref{WVP1}, we switch variables to $x\equiv L/R_v T$, which gives
	\beqn
		\int^T_{\Ttp} \frac{e^{-L/R_v T'}}{T'}\; dT' \ = \ \int_x^{x_\mathrm{tp}} \frac{e^{-x'}}{x'} dx' \ \approx\ \frac{1}{x_{\mathrm{av}}}\int_x^{x_\mathrm{tp}} e^{-x'} dx' \ \approx\ \frac{e^{-x}}{x_{\mathrm{av}}}.
		\label{int_approx}
	\eeqn
To make the first approximation above we note that $x$ varies between -18 and -27 as $T$ varies between $T=300$ K and $\Ttp \equiv  200$ K, and so the exponential $e^{-x'}$ is varying much more than the factor of $1/x'$. We thus set the latter  to $1/x_{\mathrm{av}}\equiv R_v \Tav/L$ where $\Tav \equiv (\Ts + \Ttp)/2$ and \Ts\ is the surface temperature, and pull $1/x_{\mathrm{av}}$ outside the integral. The last approximation follows from neglecting the boundary term $e^{-x_\mathrm{tp}}$. Plugging \eqnref{int_approx} into \eqnref{WVP1} then yields Eqn. \eqnref{WVP2}. 
 
 	As a  check, Figure \ref{wvp_check} juxtaposes \WVP\ diagnosed from a CRM RCE simulation with \Ts=300 K with \WVP\ estimated via \eqnref{WVP2}, with \RH\ set to 70\% for the latter by eyeball fit and \gammaav\ diagnosed via \eqnref{gamma_diag}. The constant \RH\ assumption underestimates boundary layer \WVP, and neglect of the ice phase (and concomitant lower saturation vapor pressure) leads to a slight overestimate of WVP at upper levels, but otherwise the fit is decent.

	\section{Spectral integral of $-\ppt F_k$}
	To evaluate the spectral integral in \eqnref{cts1}, we first restrict the integral to the  $(0,1000\ \cminverse)$ water vapor rotational band in which most radiative cooling occurs.  Fig. 4.19 of \pierre, reproduced in Fig. \ref{kappa_h2o}, shows that the median absorption coefficient $\kappa(k)$ in this band (as computed in sub-bands of width 50 \cminverse) can be roughly parameterized as 
		\beqn
			\kappa(k) \approx \kappa_0 \exp(-k/l_k), \quad \mbox{where $l_k \approx 65\ \cminverse$ .}
		\label{kappa_param}
		\eeqn
		We do not specify a precise value for $\kappa_0$, but we do specify the `scale wavenumber' $l_k$ as this parameter exerts a strong influence  on the magnitude of LW flux divergence, as we'll see momentarily. Now, the values of $\kappa(k)$ in \eqnref{kappa_param} span six orders of magnitude, and hence so does \tauk\ at a fixed $T$. Furthermore, for the tropospheric temperatures we consider, the spectral range of $\tauk(T)$ is essentially $(0,\infty)$, and in particular always includes \tauk=1. Let us denote the wavenumber at this peak as $k_1(T)$, i.e.  $\tau_{k_1(T)}=1$.  Since the function $\tauk \exp(-\tauk)$ has an integral of 1 across $(0,\infty)$ with a peak  at $\tauk=1$,  we  approximate it as a Dirac delta function $\delta(\tauk- 1)$. Transforming to $k$-coordinates then yields
       \begin{align}
                \tau_k e^{-\tau_k} & \approx \delta(\tau_k- 1)  \n \\
                                            & = \left|\partialderf{\tau_k}{k}(k_1)\right|\inverse\delta(k-k_1) \n  \\
                                            & =  \left|\partialderf{\ln\tau_k}{k}(k_1)\right|\inverse\delta(k-k_1) \n  \\ 
                                            & =  \left|\der{\ln\kappa}{k}(k_1)\right|\inverse\delta(k-k_1) \n \\
                                            & = l_k\,  \delta(k-k_1) \n \label{delta_approx}
        \end{align}
where the first equality is the standard chain rule for the delta function, the second equality follows from $\tau_{k_1}(T)=1$, the third from \eqnref{tau2}, and the fourth from \eqnref{kappa_param}. Plugging this into \eqnref{cts1}, evaluating the (now trivial) integral, and making the identification 
	\beqn
		\Delta k/e = l_k
		\n
		\label{Deltak}
	\eeqn
	then yields \eqnref{cts2}. Note that $l_k$ sets the width $\Delta k$ of the spectral region at a given height that exhibits significant cooling to space.

%Though a complete analysis relating the \Ts-dependence of the spectrally resolved expression \eqnref{ppzfk} to the \Ts-dependence of the bulk fluxes $U$ and $D$ would be involved, we can infer that the decrease of $\ppz F(T)$ with \Ts\ is due to the pressure broadening factor $p/p_0$ in \eqnref{ppzfk}, as follows. From \eqnref{pressure} we have
%	\beqn
%		\frac{ d \ln  p}{d \Ts} = -\frac{g}{R_d \gammaav}\left[ \frac{1}{\Ts} + \frac{1}{\gammaav}\der{\gammaav}{\Ts}\ln(T/\Ts) \right] \ .
%		\n
%	\eeqn 
%	This varies between roughly $-1.5 \%\ \Kinverse$ in the lower troposphere to $-3 \%\ \Kinverse$ in the upper troposphere, larger in magnitude than the observed change of  $\sim -0.5 \%\ \Kinverse$ in $\ppz F(T)$. Since the only other \Ts\ dependence in Eqn. \eqnref{ppzfk} is through $-U_k - D_k$, the \Ts-dependence of the latter must offset that due to pressure broadening, but with pressure broadening dominating.


	
%========%
% Figures    %
%========%
\pagebreak

%Figure rhov_fig
\begin{figure}[h]
	\begin{center}
			\includegraphics[scale=0.7]{../figures/rhov.pdf}
		\caption{Profiles of \rhov\ from our RCE simulations at various SST, plotted using temperature as a vertical coordinate and with both linear and log scales. When plotted this way, the \rhov\ profiles at different \Ts\ collapse onto a single curve, i.e. $\rhov(T)$ has no \Ts\ dependence. 
		\label{rhov_fig}
		}
	\end{center}
\end{figure}


%Figure tauk
\begin{figure}[h]
	\begin{center}
			\includegraphics[scale=0.7]{../figures/tauk.pdf}
		\caption{\textbf{Left}: Profiles of $\tau_k(T)$ for $k=$ 100, 500, and 900 \cminverse\ as diagnosed from \eqnref{tau1} (solid) and \eqnref{tau2} (dashed), respectively. \textbf{Right}: Profiles of $\tau_k(T)$ from \eqnref{tau2} for $k=$ 100, 500, and 900 \cminverse\ and for \Ts=(280,\ 290,\ 300,\ 310) K (cold to warm colors).
		\label{tauk}
		}
	\end{center}
\end{figure}

%%Figure UD_tinv
%\begin{figure}[h]
%	\begin{center}
%			\includegraphics[scale=0.6]{../figures/UD_tinv.pdf}
%		\caption{Upwelling and downwelling LW fluxes $U$ and $D$  as diagnosed from CRM RCE simulations at \Ts=(280,\ 290,\ 300,\ 310) K. These fluxes are plotted in height, pressure, and temperature coordinates.
%		\label{UD_tinv}
%		}
%	\end{center}
%\end{figure}

%Figure pptflw_tinv_dam
\begin{figure}[h]
	\begin{center}
			\includegraphics[scale=0.6]{../figures/pptflw_tinv_dam.pdf}
		\caption{LW flux divergence in temperature coordinates $-\ppt F$, as diagnosed from CRM RCE simulations at \Ts=(280,\ 290,\ 300,\ 310) K. These fluxes are plotted in height, pressure, and temperature coordinates to emphasize the \Ts-invariance of  $(-\ppt F)(T)$.
		\label{pptflw_tinv_dam}
		}
	\end{center}
\end{figure}

%Figure pptfsw_tinv_dam
\begin{figure}[h]
	\begin{center}
			\includegraphics[scale=0.6]{../figures/pptfsw_tinv_dam.pdf}
		\caption{As in Fig. \ref{pptflw_tinv_dam}, but for SW instead of LW.
		\label{pptfsw_tinv_dam}
		}
	\end{center}
\end{figure}

%Figure Qnet_varsst
\begin{figure}[h]
	\begin{center}
			\includegraphics[scale=0.6]{../figures/Qnet_varsst.pdf}
		\caption{Column-integrated cooling $Q$ vs. \Ts\ (black circles), along with slopes $d Q/d \Ts$ (red lines) as diagnosed from \eqnref{dqdts}. These are shown for the SW (left), LW (center) and net (right) channels.  The black dashed lines connect the black circles and give a benchmark slope against which to compare the red lines. The `net' panel also gives CRM-diagnosed precipitation values in blue stars.
		\label{Qnet_varsst}
		}
	\end{center}
\end{figure}


%Figure wvp_check
\begin{figure}[h]
	\begin{center}
			\includegraphics[scale=0.8]{../figures/wvp_check.pdf}
		\caption{Comparison of \WVP\ as diagnosed from DAM (solid) to that estimated by \eqnref{WVP2} (dashed), in both linear and log coordinates.
		\label{wvp_check}
		}
	\end{center}
\end{figure}

 %Figure kappa_h2o
\begin{figure}[h]
        \begin{center}
                        \includegraphics[scale=1.75]{../figures/kappa_h2o.pdf}
                \caption{Minimum, maximum, median, and 25th and 75th percentile 
values of $\kappa$ for \htwo\ for 50 \cminverse\ intervals. Pulled from \cite{pierrehumbert2010}.
                \label{kappa_h2o}
                }
        \end{center}
\end{figure}

%%Figure trop_deep_cartoon
%\begin{figure}[h]
%	\begin{center}
%			\includegraphics[scale=0.65]{../figures/trop_deep_cartoon.pdf}
%		\caption{Cartoon depicting the two contributions to the increase in $Q$ with \Ts\ in Eqn. \eqnref{dqdts}. Increasing the temperature range of the troposphere by $\Delta\Ts$ increases its depth by $\Delta \Ts/\Gamma$, and also  decreases the lapse rate, causing an additional (and comparable) increase in the depth of the troposphere.   
%		\label{trop_deep_cartoon}
%		}
%	\end{center}
%\end{figure}



%%Figure sw_tinv
%\begin{figure}[h]
%	\begin{center}
%			\includegraphics[scale=0.6]{../figures/sw_tinv.pdf}
%		\caption{As in Fig. \ref{UD_tinv}, but for SW radiances \USW\ and \DSW, as well as flux \FSW.
%		\label{sw_tinv}
%		}
%	\end{center}
%\end{figure}
%
%%Figure ppzfnet_tinv
%\begin{figure}[h]
%	\begin{center}
%			\includegraphics[scale=0.8]{../figures/ppzfnet_tinv.pdf}
%		\caption{Shortwave, longwave, and net flux divergences as diagnosed from our various SST simulations. A rough \Ts-invariance is evident for $\ppz\FSW$ as for $\ppz \FLW$, but the spreads are opposite, yielding a $\ppz \Fnet$ profile that is strongly \Ts-invariant.
%		\label{ppzfnet_tinv}
%		}
%	\end{center}
%\end{figure}

\bibliographystyle{apa}
\bibliography{/Users/climateloaner/Dropbox/bibtex_mendeley/library}
%\bibliography{/Users/nadir/Dropbox/bibtex_mendeley/library}


\end{document}

