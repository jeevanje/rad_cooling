%% Draft settings
\documentclass[10pt]{article}
\usepackage{amsmath}
\usepackage{amssymb}
\usepackage{graphicx}
\usepackage{subfigure}
\usepackage{color}
 \usepackage{lineno}
\usepackage{simplemargins}
\usepackage{natbib}

% \linenumbers*[1]
 \usepackage[T1]{fontenc} % For citing barki{\dj}ija
\setkeys{Gin}{draft=false}


% Margins
\setleftmargin{1in}
\setrightmargin{1in}
\setbottommargin{1in}
\settopmargin{1in}
 
% My Commands
%\input{tensor_book_shrtcts.tex}
\newcommand{\comment}[1]{\textcolor{blue}{[{#1}]}}


%%Nadir's Shortcuts
\newcommand{\beqn}{\begin{equation}}
\newcommand{\eeqn}{\end{equation}}
\newcommand{\beqa}{\begin{eqnarray}}
\newcommand{\eeqa}{\end{eqnarray}}
\newcommand{\beqanonum}{\begin{eqnarray*}}
\newcommand{\eeqanonum}{\end{eqnarray*}}
\newcommand{\beqnonum}{\begin{equation*}}
\newcommand{\eeqnonum}{\end{equation*}}
\newcommand{\jump}{\vspace{0.5cm}}
\newcommand{\bbf}{\begin{bf}}
\newcommand{\ebf}{\end{bf}}
\newcommand{\eqnref}[1]{(\ref{#1})}
\newcommand{\defn}[1]{\begin{bf}\emph{#1}\end{bf}}
\newcommand{\reals}{\ensuremath{\mathbb{R}}}
\newcommand{\complex}{\ensuremath{\mathbb{C}}}
\newcommand{\integers}{\ensuremath{\mathbb{Z}}}
\newcommand{\half}{\ensuremath{\frac{1}{2}}}
\newcommand{\n}{\nonumber}
\newcommand{\inverse}{^{-1}}

%calculus shorthand
\newcommand{\timeder}{\frac{d}{dt}}
\newcommand{\partialder}[1]{\frac{\partial}{\partial #1}}
\newcommand{\partialderf}[2]{\ensuremath{\frac{\partial #1}{\partial #2}}}
\newcommand{\der}[2]{\ensuremath{\frac{d #1}{d #2}}}
\newcommand{\dx}{\ensuremath{\frac{d}{dx}}}
\newcommand{\ddx}{\ensuremath{\frac{d}{dx}}}
\newcommand{\kvec}{\ensuremath{\vec{k}}}
\newcommand{\uvec}{\ensuremath{\mathbf{u}}}
\newcommand{\zhat}{\ensuremath{\mathbf{\hat{z}}}}
\newcommand{\khat}{\ensuremath{\mathbf{\hat{k}}}}
\newcommand{\unitvect}[1]{\ensuremath{\mathbf{\hat{#1}}}}
\newcommand{\ppx}{\ensuremath{\partial_x}}
\newcommand{\ppy}{\ensuremath{\partial_y}}
\newcommand{\ppz}{\ensuremath{\partial_z}}
\newcommand{\ppp}{\ensuremath{\partial_p}}


% radiation shorthand
\newcommand{\cotwo}{\ensuremath{\mathrm{CO_2}}}
\newcommand{\htwo}{\ensuremath{\mathrm{H_2O}}}
\newcommand{\QLW}{\ensuremath{Q_\mathrm{LW}}}
\newcommand{\QSW}{\ensuremath{Q_\mathrm{SW}}}
\newcommand{\Qnet}{\ensuremath{Q_\mathrm{net}}}
\newcommand{\FLW}{\ensuremath{F}}
\newcommand{\FSW}{\ensuremath{F^\mathrm{SW}}}
\newcommand{\Fnet}{\ensuremath{F^\mathrm{net}}}
\newcommand{\olr}{\ensuremath{\mathrm{OLR}}}
\newcommand{\OLR}{\ensuremath{\mathrm{OLR}}}
\newcommand{\trans}{\ensuremath{\mathcal{T}}}
\newcommand{\cool}{\ensuremath{\mathcal{C}}}
\newcommand{\cminverse}{\ensuremath{\mathrm{cm^{-1}}}}
\newcommand{\pierre}{P10}

% meteorology shorthand
\newcommand{\qv}{\ensuremath{q}}
\newcommand{\rhov}{\ensuremath{\rho_\mathrm{v}}}
\newcommand{\Hv}{\ensuremath{H_\mathrm{v}}}
\newcommand{\Rv}{\ensuremath{R_\mathrm{v}}}
\newcommand{\qa}{\ensuremath{q_a}}
\newcommand{\qvstar}{\ensuremath{q^*}}
\newcommand{\Ta}{\ensuremath{T_a}}
\newcommand{\Tav}{\ensuremath{T_\mathrm{av}}}
\newcommand{\Ts}{\ensuremath{T_\mathrm{s}}}
\newcommand{\ps}{\ensuremath{p_s}}
\newcommand{\RH}{\ensuremath{\mathrm{RH}}}
\newcommand{\WVP}{\ensuremath{\mathrm{WVP}}}
\newcommand{\ztop}{\ensuremath{z_\mathrm{top}}}
\newcommand{\ztp}{\ensuremath{z_\mathrm{tp}}}
\newcommand{\zlcl}{\ensuremath{z_\mathrm{LCL}}}
\newcommand{\Tlcl}{\ensuremath{T_\mathrm{LCL}}}
\newcommand{\Ttp}{\ensuremath{T_\mathrm{tp}}}
\newcommand{\lapseav}{\ensuremath{\Gamma_\mathrm{av}}}
\newcommand{\gammaav}{\ensuremath{\Gamma_\mathrm{av}}}
\newcommand{\Kinverse}{\ensuremath{\mathrm{K^{-1}}}}
\newcommand{\Htauk}{\ensuremath{H_{\tau_k}}}

%Variables
\newcommand{\figurepath}{../figures/}


\begin{document}

%% ------------------------------------------------------------------------ %%
%
%  TITLE
%
%% ------------------------------------------------------------------------ %%


\title{An Analytical Model for  Radiative Cooling in a Moist Atmosphere}

%% ------------------------------------------------------------------------ %%
%
%  AUTHORS AND AFFILIATIONS
%
%% ------------------------------------------------------------------------ %%


 \author{Nadir Jeevanjee\footnote{Department of Physics, University of California at Berkeley, Berkeley, CA 94702  USA. jeevanje@berkeley.edu (corresponding author)} \footnote{Climate and Ecosystems Science Division, Lawrence Berkeley National Laboratory, Berkeley, CA USA} and David Romps\footnote{Department of Earth and Planetary Sciences, University of California at Berkeley, Berkeley, CA 94702  USA.} \footnote{Climate and Ecosystems Science Division, Lawrence Berkeley National Laboratory, Berkeley, CA USA}
}

\maketitle

\begin{abstract}
Abstract here.

%\vspace{0.5cm}
%
%
\end{abstract}


%% ------------------------------------------------------------------------ %%
%
%  TEXT
%
%% ------------------------------------------------------------------------ %%


\section {Introduction}
Introduction here

%================%
% Model construction  %
%================%
\section{Construction of the model}
In this section we construct a spectrally resolved, analytical model for radiative cooling due to water vapor in an Earth-like atmosphere where water vapor is a condensable, minor constituent. The basic strategy will be to  simply parameterize $\kappa(k)$, the mass absorption coefficient ($\mathrm{m^2/kg}$) of water vapor as a function of wavenumber (or inverse wavelength) $k$, and use this parameterization to compute the optical depth $\tau_k$. This will give us the transmission function $\trans_k \equiv \exp(-\tau_k)$, which we can use to estimate the longwave flux divergence $\ppz \FLW_k$ via the cooling-to-space (CTS) approximation
	\beqn
		\ppz \FLW_k \approx \pi B_k \frac{d \trans_k}{dz} 
		%\quad\quad \quad \mbox{(cooling-to-space approx)}   
	\label{cts}
	\eeqn
where $B_k$ is the planck density with respect to wavenumber. Spectral integration of the above will then yield the total longwave radiative cooling $\ppz \FLW$, in $\mathrm{W/m^3}$.

To parametrize $\kappa(k)$ for \htwo, we begin with a plot of the minimum, maximum, median, and 25th and 75th percentile values of $\kappa$ for 10 \cminverse\ intervals, pulled from \cite{pierrehumbert2010} (hereafter \pierre). If we take the median values as representative of $\kappa(k)$ in a given 10 \cminverse\ interval, then the roughly
linear behavior of its logarithm suggests that we might parameterize $\ln(\kappa(k))$  to first order as a piecewise linear functions whose slopes and intercepts match those of the median curve in Fig. \ref{kappa_h2o}. This yields
\beqn
	\ln \kappa(k) = \left\{ \begin{array}{lc}
						\ln 10 + (\ln 10^{-5} -\ln 10)\frac{k-100}{1000-100} & 100 < k < 1000 \ \cminverse \\
							& \\
						\ln 10^{-5} + (\ln 1 -\ln 10^{-5})\frac{k-100}{1000-100} & 1000 < k < 1500 \ \cminverse \\
					\end{array} \right.								
\label{kappa_param}
\eeqn
Given $\kappa(k)$, the optical depth $\tau_k$ at height $z$ is given by 
	\beqn
		\tau_k = \int_z^\infty \kappa(k) \frac{p}{p_0}S_k(T) \rho_v\, dz'  \label{tau1}
	\eeqn
where $\rho_v$ is the density of water vapor and $p/p_0$ and $S_k(T)$ represent the effects of pressure and temperature scaling, respectively (\pierre). The reference pressure and temperature $p_0$ and $T_0$ are set to 100 hPa and 260 K as in \pierre, and the temperature scaling  $S_k(T) =  \exp[-T_k^*(1/T-1/T_0)]$, where $T_k^*$ may be estimated from Fig. 4.20 of \pierre. 

As a first sanity check on our approach we diagnose $\tau_k$ using Eqn. \eqnref{tau1} and output from an RCE simulation at surface temperature $\Ts=300$ K with \htwo\ as the only greenhouse gas .  We then use this to numerically compute the weight function $\ppz\trans_k$ via finite-difference, and then use \eqnref{cts} to compute the flux divergence $\ppz \FLW_k$. Finally, we  compute the spectral cooling rate $\cool_k$ as 
	\beqn
		\cool_k  = \frac{\ppz \FLW_k}{\rho C_p} \; .   \label{cooling}
	\eeqn
This cooling is plotted in the left-hand panel of Figure \ref{cooling}, and  may be compared to the line-by-line (LBL) calculation of \cite{huang2013}, reproduced in Figure \ref{lbl_cooling}. Apart from the absence of cooling in the  \cotwo\ band around 660 \cminverse, and the enhanced cooling in the water vapor window 800-1200 \cminverse due to continuum absorption, our approximate calculation is in decent qualitative and quantitative agreement with the LBL calculation. The main feature of both is  the diagonal band of cooling running from $(k,T) \approx (200\ \mathrm{K} ,200 \cminverse) $ to $(k,T) \approx (600\ \mathrm{K},\ 280\  \cminverse)$. As is well-known \citep[e.g.][]{goldblatt2013}, this band represents cooling (at each $k$) where  $\tau_k\approx 1$ \citep[e.g.,][]{wallace2006}, and this level descends with increasing $k$ because $\kappa$ decreases with $k$, requiring a larger water vapor path (and hence a lower altitude) to reach $\tau_k=1$. The linearity of the band in $(k,T)$ space is due to the common exponential dependence of both $\kappa(k)$ and $\tau_k(T)$, as will be evident and further discussed below.

As a further, independent sanity check we may numerically integrate our diagnosed $\ppz \FLW_k$ over $k$ space and compare the resulting $\ppz F$ profile to that output by the RRTM radiation scheme coupled to our RCE simulation. These profiles are shown in Figure
\ref{ppzf} in magenta and black, respectively. Our estimate of $\ppz \FLW$ is biased low in general; we speculate that this is due (in the lower troposphere) to our omission of continuum effects, and (in the far upper troposphere) our use of the median values for $\kappa$, which ignores the highest $\kappa$ values which are generating the cooling in that region. Nevertheless, our estimate captures the overall shape and magnitude of the benchmark RRTM curve reasonably well.

That the basic features of $\cool_k$  and $\ppz \FLW$ survive the approximations \eqnref{cts} and \eqnref{kappa_param} is heartening, but the expression \eqnref{tau1} for $\tau_k$ is still too complicated to be analytically tractable. We thus make several further approximations. First, we define an average tropospheric temperature 
\beqn
	\Tav \equiv (\Ts + \Ttp)/2
	\n
\eeqn
 where $\Ttp = 200$ K is an approximate tropopause temperature. We  then approximate the pressure as 
	\beqn 
		p \approx \ps \exp(-z/H)\quad \quad \mbox{where $H \equiv  R_d \Tav/g$} \; .
	\eeqn  
	We can also Taylor expand $S_k(T)$ and $\rhov(T) = \RH\, p_v^*(T)/R_ vT$,  where $p_v^* = p_v^\infty \exp(-L/R_vT)$ is the vapor pressure of water ($p_v^\infty = 2.5\times 10^{11} $ Pa), around this \Tav. With these simplifications  and the assumption of a constant average lapse rate \gammaav\ so that $T=\Ts-\gammaav z$, Eqn. \eqnref{tau1} becomes tractable and we have	
	\beqn
		\tau_k(T) = \kappa(k)\Htauk \underbrace{\left[\rhov(\Tav)\exp\left(\frac{L}{R_v\Tav^2}(T-\Tav)\right)\right]}_{\mbox{\htwo\ C-C scaling}}\
							  \underbrace{\left[\frac{\ps}{p_0}\exp\left(-\frac{\Ts-T}{\gammaav H}\right)\right]}_{\mbox{ $p$-broadening}}\
							  \underbrace{\left[S_k(\Tav)\exp\left(\frac{T^*_k}{\Tav}(T-\Tav)\right)\right]}_{\mbox{$T$-scaling}}
	\label{tau2}
	\eeqn
where 
	\beqn
		1/\Htauk \equiv \der{\ln \tau_k}{z} = \frac{L\gammaav}{R_v \Tav^2}+ \frac{1}{H} + \frac{T_k^*\gammaav}{\Tav^2}  
	\eeqn
		is a slightly $k$-dependent but $z$-independent scale height for $\tau_k$, with contributions from C-C scaling, pressure broadening, and temperature scaling respectively, as in Eqn. \eqnref{tau2}. Note that though these expressions are approximate, they contain no unobservable, tunable parameters. 		
		
		An analytical expression for the weight function then immediately follows as 
	\beqn
		\ppz \trans_k = \frac{1}{\Htauk} \tau_k e^{-\tau_k}  \ .
	\label{weight}
	\eeqn
 Combining this with the CTS approximation \eqnref{cts}  yields the cooling rate shown in the right panel of Figure \ref{cooling}, as well as the green profile in Fig. \ref{ppzf}.  Though many approximations have been made, we do not find  a major quantitative degradation relative to the cooling derived from \eqnref{tau1}, and the qualitative agreement with both the LBL (Fig. \ref{lbl_cooling}) and RRTM benchmarks is still good.

Now that we have some confidence that our approximations have retained the essential, spectrally-resolved physics,  we may now approach our goal, which is  an analytic expression for the total longwave cooling. This will require an analytic integration over $k$ space. We begin by noting that the $k$-dependence of $\ppz \trans_k$ is almost entirely through the function $\tau_k e^{-\tau_k}$ (the $k$-dependence of \Htauk\ is weak, $\pm 10 \%$ of the mean over our $k$-range).  The function $\tau_k e^{-\tau_k}$(as a function of $\tau_k$) has an integral of 1 across $(0,\infty)$, with a peak at $\tau_k=1$, so we may approximate it as a Dirac delta function peaked at $\tau_k=1$. We can then switch variables to $k$, where $T$ is held fixed and the peak of $\tau_k e^{-\tau_k}$ is located where	$\tau_{k}(T)=1$. We denote this wavenumber as $k_1(T)$, which by  \eqnref{tau2} is given by 
	\beqn
		k_1(T) =  l_k\left\{\ \ln[ \WVP_0\kappa(0)] + \ln(\ps/p_0) - \frac{g}{R_d\,\Gamma}\ln(\Ts/T) - \frac{L}{R_vT}\  \right\} \; 
	\label{k1}
	\eeqn
	where $l_k \equiv  (\der{\ln\kappa}{k})^{-1} = 65\ \cminverse$, derived from \eqnref{kappa_param}. Putting this together yields
	\begin{align}
		\tau_k e^{-\tau_k} & \approx \delta(\tau_k- 1)  \n \\
					    & = \left|\partialderf{\tau_k}{k}(k_1)\right|\inverse\delta(k-k_1) \n  \\
					    & =  \left|\partialderf{\ln\tau_k}{k}(k_1)\right|\inverse\delta(k-k_1) \n  \\ 
					    & =  \left|\der{\ln\kappa}{k}(k_1)\right|\inverse\delta(k-k_1) \n \\
					    & = l_k\,  \delta(k-k_1) \label{delta_approx}
	\end{align}
where the first equality is the standard chain rule for the delta function, the second equality follows from $\tau_{k_1}(T)=1$, the third from \eqnref{tau2}, and the fourth from our definition of  $l_k$.

We can now plug  \eqnref{delta_approx} into \eqnref{weight} and plug that into the  CTS approximation \eqnref{cts}. The $k$ integral is then trivial and yields the desired analytical expression for the longwave flux divergence $\ppz F$, 
	\beqn
		\ppz \FLW = \pi B_{k_1}(T)\left(\frac{L \, \Gamma}{R_vT^2}+\frac{g}{R_d T} \right)l_k \; .
	\label{ppzf_eqn}
	\eeqn
This expression is a central result of this paper, and is plotted in blue in Fig. \ref{ppzf}. There is some further quantitative degradation relative to our previous approximations, and these admittedly add up to a roughly factor of two error over most of the troposphere (and worse near the tropopause) relative to the RRTM benchmark. However, our procedure of incremental approximation and comparison suggests that \eqnref{ppzf_eqn} does capture the first-order physics of radiative cooling from water vapor, and that its qualitative agreement with benchmarks is not coincidental. We can thus proceed to use it to gain insight into this basic atmospheric process.

\section{Basic properties of $\ppz \FLW$}
	Let us begin by interpreting the factors in \eqnref{ppzf_eqn}. To do this, note that with the CTS approximation \eqnref{cts} and the observation that cooling at a temperature $T$ occurs in a $k$-space interval $\Delta k$ around $k_1$, we can write 
	\begin{align}
		\ppz \FLW & =  \pi B_{k_1}(T) \ppz \trans_{k_1} \Delta k \n \\
				 & =  \pi B_{k_1}(T) \ppz(\ln \tau_{k_1}) \, \tau_{k_1}e^{-\tau_{k_1}} \Delta k \; . 
					\label{ppzf_eqn_heur}
	\end{align}
The logarithmic derivative $\ppz (\ln \tau_{k_1})$ factor is just the $(L\Gamma/R_vT^2+g/R_d T )$ factor appearing in \eqnref{ppzf_eqn}, and represents a temperature dependent (but $k$-independent) inverse scale height for optical depth. The  two terms represent the influence of  Clausius-Clapeyron (C-C) scaling of \rhov\ and pressure broadening, respectively. The factor $\tau_{k_1}e^{-\tau_{k_1}} \Delta k = \Delta k/e$ in \eqnref{ppzf_eqn_heur} is just $l_k$, and our value of $l_k = 65$ \cminverse\ gives $\Delta k \approx 200\ \cminverse$, in good eyeball agreement with Fig. \ref{cooling}.

Next let us ask why $\ppz F$ increases with $T$. From \eqnref{ppzf_eqn} we see that the C-C scaling of \rhov\ and pressure broadening actually serve to \emph{decrease} $\ppz F$ as $T$ increases. Even though both processes serve to increase $\tau$ exponentially with $T$, the \emph{logarithmic} derivative decreases with $T$. Furthermore, this effect is not necessarily  reversed by the explicit $T$-dependence of $B_{k_1}(T)$. This can be seen in Figure \ref{ppzf_phase}, where we plot \eqnref{ppzf_eqn} but for arbitrary $k$ as well as $T$. The black dashed line is  $k_1(T)$ from \eqnref{k1}. One can see that $\ppz F$ increases with $T$ along  $k_1(T)$, but that if $k_1(T)$ were flatter (e.g. $k_{\mathrm{alt}}(T)$ in Fig. \eqnref{ppzf_phase}), then $\ppz F$ would no longer be monotonic in $T$. Thus the monotonic increase of $\ppz F$ with $T$ is not an inalterable feature of radiative transfer, but rather an accident of water vapor spectroscopy.

%===========%
% T-invariance  %
%===========%
\section{Temperature invariance of  $\ppz \FLW$} 
Another immediate consequence of \eqnref{ppzf_eqn} is that $\ppz F$ should to a good approximation be a function of $T$ alone, with little pressure dependence. This is evident in Fig. \ref{ppzf_tinv_theory}, where we plot profiles of $\ppz F$ derived from \eqnref{ppzf_eqn} for $\Ts=(280,290,300,310)$ K, with $T=T_s-\Gamma z$ for the left-most panel and $p=\ps(T/\Ts)^{(g/R_d\Gamma)}$ for the center panel. One can see that in temperature coordinates, the profiles of $\ppz F$ for various \Ts\ collapse almost perfectly onto a single curve. The same behavior is exhibited by the benchmark $\ppz F$ profiles generated via RRTM from four CRM simulations of RCE with the same set of \Ts, as shown in Figure \ref{ppzf_tinv_dam}. \comment{These have \cotwo\ though, and must be re-run without for production runs.}

The simple behavior of $\ppz F$ in temperature coordinates allows for a simple explanation of the result that column-integrated radiative cooling (and hence precipitation) changes at a rate of roughly 1-3\% \Kinverse\ with warming \citep{ogorman2012}. This result is known to be a result of radiative water-vapor feedbacks \citep{pendergrass2014}, but a back-of-the-envelope estimate has been lacking. To obtain this, we first note that the temperature invariance of radiative cooling still holds when extended to net (shortwave plus longwave cooling), as evident in Fig. \ref{ppzf_net}, which  shows free-tropospheric (i.e. above the lifting condensation level $\zlcl  \approx 500\ \mathrm{m}$) profiles of shortwave flux divergence  as well as net flux divergence $\ppz \Fnet$, as diagnosed by RRTM for our variable-SST RCE simulations. We can then write the net, column-integrated free-tropospheric radiative cooling $Q$ as 
	\beqn
		Q = \int_{\zlcl}^{\ztp} \partial_{z'} \Fnet dz'  = \int_{\Ttp}^{\Tlcl} \frac{\partial_{z'} \Fnet}{\Gamma} dT' \n
	\eeqn
where we no longer assume a vertically constant lapse rate $\Gamma$.  The fractional change in $Q$ with surface temperature evaluated  is then just
	\beqn
		\der{\ln Q}{\Ts}	\approx \der{\ln Q}{\Tlcl} = \frac{1}{Q} \left. \frac{\ppz \Fnet}{\Gamma}\right |_{\mathrm{LCL}}.   \n
	\eeqn
Evaluating this at $\Ts=300$ K with typical values $Q=100 \ \mathrm{W/m^2}$, $\ppz \Fnet(\Tlcl)=0.01 \ \mathrm{W/m^3}$, and $\Gamma(\Tlcl) = .005\ \mathrm{K/m}$ yields 2 \% \Kinverse, in excellent agreement with range of observed values reported in the literature. 


%==================%
% runaway greenhouse  %
%=================%
\section{The local runaway greenhouse} 
There is a close connection between our CTS approximation for  $\ppz F$ in Eqn. \eqnref{ppzf_eqn} and outgoing longwave radiation (OLR), as the latter may be written without approximation as a surface term plus a cooling-to-space term:
	\begin{align}
		\OLR   & = \int \pi B_k(0)\exp(-\tau_k(0))\, dk \quad + \quad \int dk \int_0^\infty \pi B_k(\partial_{z} \trans_k ) \, dz \;    \n  \\
			   &  = \int \pi  B_k(\Ts)\exp(-\tau_k(\Ts))\, dk \quad + 
			   		\quad \int_{\Ttp}^{\Ts} \pi B_{k_1(T)}\left(\frac{L }{R_vT^{2}}+\frac{g}{R_d T} \right)\, \frac{l_k}{\Gamma}\, dT  	
				   \label{olr}
	\end{align}
where we switched from $z$ to $T$ coordinates in the second line. The local runaway greenhouse \citep{pierrehumbert1995} occurs when $\Ts$ is large enough that  much of the thermal infrared has $\tau_k(\Ts) \gg 1$ (e.g. Figs. \ref{cooling} and  \ref{lbl_cooling}) and so the surface contribution is small, and $\ppz F$ is a function of $T$ only, and thus both terms in \eqnref{olr} (and hence the \OLR\ itself) are insensitive to changes in \Ts. To what degree is such a local runaway evident in our RCE simulations?

The left panel of Figure \ref{fluxes} plots profiles of upwelling longwave $U$ generated by RRTM for our various-SST runs. An increase in \OLR\ is evident throughout this SST range, and for $\Ts< 300 $ K this is clearly due in large part to increasing surface emission. For $\Ts =300$ and 310 K, however,  $U(300\ \mathrm{K})$ is virtually identical, so it is not the surface term that is producing the additional \OLR in the $\Ts > 300 $ K regime. (Note that this is roughly consistent with the results of \cite{goldblatt2013}, who find that surface emission becomes irrelevant for $\Ts >310$ K.)   Thus, it must be the 2nd term in \eqnref{olr} producing the additional OLR . But, we don't expect the $\Ts$ in the upper limit of the integral to contribute as emission from layers warmer than $300$ K or so do not seem to contribute to $U$ at lower temperatures, and Fig. \ref{ppzf_tinv_theory}
shows that the integrand itself is a fixed function of temperature, at least so far as the lapse rate $\Gamma$ is held fixed. This suggests that the increase in \OLR\ is due to changes in $\Gamma$, which decreases by roughly 1 K/km throughout the troposphere between the $\Ts = 300$ and 310 K runs (Fig. \ref{lapse_rates}).

To test the hypothesis that changes in $\Gamma$ are responsible for the increase in $F(T)$ and hence OLR between these runs, we construct two profiles of $F$ for  $\Ts=300$ K using Eqns. \eqnref{flux}, \eqnref{tau2}, and \eqnref{ppzf_eqn}, where the only difference in the two profiles are the values of $\Gamma$ used in the integral in \eqnref{flux} (we take these to be $\Gamma =5,\ 6$ K/km). The various


	
%========%
% Figures    %
%========%
\pagebreak

 %Figure kappa_h2o
\begin{figure}[h]
	\begin{center}
			\includegraphics[scale=1.5]{../figures/kappa_h2o.pdf}
		\caption{Minimum, maximum, median, and 25th and 75th percentile values of $\kappa$ for \htwo\ for 10 \cminverse\ intervals. Pulled from \cite{pierrehumbert2010}
		\label{kappa_h2o}
		}
	\end{center}
\end{figure}

%Figure cooling
\begin{figure}[h]
	\begin{center}
			\includegraphics[scale=0.7]{../figures/cooling.pdf}
		\caption{Plots of cooling rate for the approximations  \eqnref{tau1} and \eqnref{tau2} .
		\label{cooling}
		}
	\end{center}
\end{figure}

%Figure lbl_cooling
\begin{figure}[h]
	\begin{center}
			\includegraphics[scale=0.4]{../figures/spectral_lw_cooling.png}
		\caption{Spectral cooling $\cool_k$, as computed by a line-by-line radiative transfer model. Taken from \cite{huang2013}.
		\label{lbl_cooling}
		}
	\end{center}
\end{figure}

%Figure ppzf
\begin{figure}[h]
	\begin{center}
			\includegraphics[scale=0.8]{../figures/ppzf.pdf}
		\caption{Profiles of $\ppz \FLW$ for our RCE simulation. The black curve is a benchmark generated by RRTM, the magenta by Eqn. \eqnref{tau1}, the green by \eqnref{tau2} with numerical integration over $k$, and the blue by direct evaluation of \eqnref{ppzf_eqn}.
		\label{ppzf}
		}
	\end{center}
\end{figure}


%Figure wvp_check
\begin{figure}[h]
	\begin{center}
			%\includegraphics[scale=0.8]{../plots/wvp_check.pdf}
		\caption{Comparison of \WVP\ as diagnosed from DAM (solid) to that estimated by \eqnref{WVP2} (dashed), in both linear and log coordinates.
		\label{wvp_check}
		}
	\end{center}
\end{figure}

%Figure ppzf_phase
\begin{figure}[h]
	\begin{center}
			%\includegraphics[scale=0.8]{../plots/ppzf_phase.pdf}
		\caption{Heat map of  $\ppz F$ evaluated via \eqnref{ppzf_eqn} but for arbitrary $k$ and $T$, with $k_1(T)$ plotted in black dashes. A fictitious alternative $k_{\mathrm{alt}}(T)$ to $k_1(T)$ is plotted in gray dashes, which would yield a non-monotonic profile of $\ppz F$, showing that the monotonicity of the latter on Earth is an accident of water-vapor spectroscopy.
		\label{ppzf_phase}
		}
	\end{center}
\end{figure}

%Figure ppzf_tinv_theory
\begin{figure}[h]
	\begin{center}
			%\includegraphics[scale=0.6]{../plots/ppzf_tinv_theory.pdf}
		\caption{Profiles of  $\ppz F$ generated from \eqnref{ppzf_eqn}  for \Ts=280,290,300, and 310 K, plotted in height, pressure, and temperature coordinates.
		\label{ppzf_tinv_theory}
		}
	\end{center}
\end{figure}

%Figure ppzf_tinv_dam
\begin{figure}[h]
	\begin{center}
			%\includegraphics[scale=0.6]{../plots/ppzf_tinv_dam.pdf}
		\caption{As in Fig. \ref{ppzf_tinv_theory}, but for simulated $\ppz F$.
		\label{ppzf_tinv_dam}
		}
	\end{center}
\end{figure}

%Figure ppzf_net
\begin{figure}[h]
	\begin{center}
			%\includegraphics[scale=0.8]{../plots/ppzf_net.pdf}
		\caption{Profiles of longwave (dashed,positive), shortwave(dashed,negative), and net (solid) flux divergences for our RCE simulation. The temperature invariance of $\ppz F$ holds for the net flux divergence as well.
		\label{ppzf_net}
		}
	\end{center}
\end{figure}

%Figure fluxes
\begin{figure}[h]
	\begin{center}
			%\includegraphics[scale=0.6]{../plots/fluxes.pdf}
		\caption{Left: Profiles of RRTM-generated longwave flux $F$ for our various-SST simulations. Center: Profile of lapse rate $\Gamma$ diagnosed directly from CRM output. Right?
		\label{fluxes}
		}
	\end{center}
\end{figure}

%Figure lapse_rates
\begin{figure}[h]
	\begin{center}
			%\includegraphics[scale=0.6]{../plots/lapse_rates.pdf}
		\caption{Plots of lapse rate $\Gamma \equiv - dT/dz$ as diagnosed directly from CRM output for our various-SST RCE simulations.
		\label{lapse_rates}
		}
	\end{center}
\end{figure}




\bibliographystyle{apa}
\bibliography{/Users/climateloaner/Dropbox/bibtex_mendeley/library}
%\bibliography{/Users/nadir/Dropbox/bibtex_mendeley/library}


\end{document}

