\documentclass[11pt]{article}

\usepackage{geometry}
\usepackage{latexsym, amsmath, amssymb}
\usepackage{psfrag}
\usepackage{mathrsfs}
\usepackage{bbm}
\usepackage[pdftex]{graphicx}
\usepackage{natbib}

\renewcommand{\thefigure}{S\arabic{figure}}

%My Standard Shortcuts - these are frequently used below!
\newcommand{\beqn}{\begin{equation}}
\newcommand{\eeqn}{\end{equation}}
\newcommand{\beqa}{\begin{eqnarray}}
\newcommand{\eeqa}{\end{eqnarray}}
\newcommand{\beqanonum}{\begin{eqnarray*}}
\newcommand{\eeqanonum}{\end{eqnarray*}}
\newcommand{\beqnonum}{\begin{equation*}}
\newcommand{\eeqnonum}{\end{equation*}}
\newcommand{\jump}{\vspace{0.5cm}}
\newcommand{\bbf}{\begin{bf}}
\newcommand{\ebf}{\end{bf}}
\newcommand{\eqnref}[1]{(\ref{#1})}
\newcommand{\defn}[1]{\begin{bf}\emph{#1}\end{bf}}
\newcommand{\reals}{\ensuremath{\mathbb{R}}}
\newcommand{\complex}{\ensuremath{\mathbb{C}}}
\newcommand{\integers}{\ensuremath{\mathbb{Z}}}
\newcommand{\half}{\ensuremath{\textstyle{\frac{1}{2}}}}
\newcommand{\comment}[1]{\textcolor{blue}{[{#1}]}}


%calculus shorthand
\newcommand{\timeder}{\frac{d}{dt}}
\newcommand{\partialder}[1]{\frac{\partial}{\partial #1}}
\newcommand{\partialderf}[2]{\ensuremath{\frac{\partial #1}{\partial #2}}}
\newcommand{\der}[2]{\ensuremath{\frac{d #1}{d #2}}}
\newcommand{\dx}{\ensuremath{\frac{d}{dx}}}
\newcommand{\ddx}{\ensuremath{\frac{d}{dx}}}
\newcommand{\kvec}{\ensuremath{\vec{k}}}
\newcommand{\uvec}{\ensuremath{\mathbf{u}}}
\newcommand{\zhat}{\ensuremath{\mathbf{\hat{z}}}}
\newcommand{\khat}{\ensuremath{\mathbf{\hat{k}}}}
\newcommand{\unitvect}[1]{\ensuremath{\mathbf{\hat{#1}}}}
\newcommand{\ppx}{\ensuremath{\partial_x}}
\newcommand{\ppy}{\ensuremath{\partial_y}}
\newcommand{\ppz}{\ensuremath{\partial_z}}
\newcommand{\ppt}{\ensuremath{\partial_T}}
%\newcommand{\ppp}{\ensuremath{\partial_p}}

% radiation shorthand
\newcommand{\cotwo}{\ensuremath{\mathrm{CO_2}}}
\newcommand{\othree}{\ensuremath{\mathrm{O_3}}}
\newcommand{\htwo}{\ensuremath{\mathrm{H_2O}}}
\newcommand{\QLW}{\ensuremath{Q_\mathrm{LW}}}
\newcommand{\QSW}{\ensuremath{Q_\mathrm{SW}}}
\newcommand{\Qnet}{\ensuremath{Q_\mathrm{net}}}
\newcommand{\FLW}{\ensuremath{F^\mathrm{LW}}}
\newcommand{\FSW}{\ensuremath{F^\mathrm{SW}}}
\newcommand{\FLWcs}{\ensuremath{F^\mathrm{LW}_{\mathrm{cs}}}}
\newcommand{\FSWcs}{\ensuremath{F^\mathrm{SW}_{\mathrm{cs}}}}
\newcommand{\USW}{\ensuremath{U^\mathrm{SW}}}
\newcommand{\DSW}{\ensuremath{D^\mathrm{SW}}}
\newcommand{\Fnet}{\ensuremath{F^\mathrm{net}}}
\newcommand{\Fnetgl}{\ensuremath{F^\mathrm{net}_{\mathrm{gl}}}}
\newcommand{\Fgl}{\ensuremath{F_{\mathrm{gl}}}}
\newcommand{\olr}{\ensuremath{\mathrm{OLR}}}
\newcommand{\OLR}{\ensuremath{\mathrm{OLR}}}
\newcommand{\trans}{\ensuremath{\mathcal{T}}}
\newcommand{\solar}{\ensuremath{I_0}}
\newcommand{\cool}{\ensuremath{\mathcal{C}}}
\newcommand{\cminverse}{\ensuremath{\mathrm{cm^{-1}}}}
\newcommand{\pierre}{P10}
\newcommand{\tauk}{\ensuremath{\tau_\lambda}}
\newcommand{\Wmsq}{\ensuremath{\mathrm{W/m^2}}}
\newcommand{\meter}{\ensuremath{\mathrm{m}}}
\newcommand{\kg}{\ensuremath{\mathrm{kg}}}
\newcommand{\Kinverse}{\ensuremath{\mathrm{K^{-1}}}}
\newcommand{\Kelvin}{\ensuremath{\mathrm{K}}}


% meteorology shorthand
\newcommand{\qv}{\ensuremath{q}}
\newcommand{\rhov}{\ensuremath{\rho_\mathrm{v}}}
\newcommand{\Hv}{\ensuremath{H_\mathrm{v}}}
\newcommand{\Rv}{\ensuremath{R_\mathrm{v}}}
\newcommand{\qa}{\ensuremath{q_a}}
\newcommand{\qvstar}{\ensuremath{q^*}}
\newcommand{\pvstar}{\ensuremath{p^*_{\mathrm{v}}}}
\newcommand{\Ta}{\ensuremath{T_a}}
\newcommand{\Tav}{\ensuremath{T_\mathrm{av}}}
\newcommand{\Ts}{\ensuremath{T_\mathrm{s}}}
\newcommand{\Tsgl}{\ensuremath{T_\mathrm{s,gl}}}
\newcommand{\ps}{\ensuremath{p_s}}
\newcommand{\RH}{\ensuremath{\mathrm{RH}}}
\newcommand{\cld}{\ensuremath{\mathrm{Cld}}}
\newcommand{\WVP}{\ensuremath{\mathrm{WVP}}}
\newcommand{\ztop}{\ensuremath{z_\mathrm{top}}}
\newcommand{\ztp}{\ensuremath{z_\mathrm{tp}}}
\newcommand{\zlcl}{\ensuremath{z_\mathrm{LCL}}}
\newcommand{\Tlcl}{\ensuremath{T_\mathrm{LCL}}}
\newcommand{\Ttp}{\ensuremath{T_\mathrm{tp}}}
\newcommand{\Text}{\ensuremath{T_\mathrm{ext}}}
\newcommand{\ptp}{\ensuremath{p_\mathrm{tp}}}
\newcommand{\lapseav}{\ensuremath{\Gamma_\mathrm{av}}}
\newcommand{\gammaav}{\ensuremath{\Gamma_\mathrm{av}}}
\newcommand{\Htauk}{\ensuremath{H_{\tau_k}}}
\newcommand{\east}{\ensuremath{\mathrm{E}}}
\newcommand{\north}{\ensuremath{\mathrm{N}}}
\newcommand{\south}{\ensuremath{\mathrm{S}}}
\newcommand{\tav}{\ensuremath{t_\mathrm{av}}}
\newcommand{\kmax}{\ensuremath{k_\mathrm{max}}}
\newcommand{\kmin}{\ensuremath{k_\mathrm{min}}}

\newcommand{\figurepath}{../../../figures/}

\begin{document}

\begin{center}
 \Large{ \bf  Supplemental Information for \emph{Mean precipitation change from a deepening troposphere} }

\jump
\large{Nadir Jeevanjee and David M. Romps}
\end{center}


%==============%
% References          %
%==============%

\bibliographystyle{apa}
\bibliography{../../../../resources/bibtex_mendeley/library}

%=======%
% Figures  %
%========%  
\pagebreak  
 
 %=====%
 % CRM %
 %=====%
 
 \pagebreak 
 %Figure crm_crh
\begin{figure}[h]
        \begin{center}
                        \includegraphics[scale=0.5]{\figurepath crm_crh.pdf}
		\caption{Snapshots  of column relative humidity (CRH) from the last day of each RCE simulation. No organization is evident, and the low CRH values associated with aggregation ($\lesssim 0.3$) are not observed. 
		\label{crm_crh}                
		}
        \end{center}
\end{figure}

%Figure pptflw_tinv
\begin{figure}[h]
        \begin{center}
                        \includegraphics[scale=0.5]{\figurepath pptflw_tinv_dam.pdf}
		\caption{LW flux divergence  $-\ppt \FLW$, as diagnosed from RRTM coupled to our CRM RCE simulations at \Ts=(280,\ 290,\ 300,\ 310) K. Fluxes are plotted from the lifting condensation level of each simulation to 22.5 km for clarity, and  in height, pressure, and temperature coordinates to emphasize the \Ts-invariance of  $(-\ppt \FLW)(T)$. 
		\label{pptflw_tinv_dam}                
		}
        \end{center}
\end{figure}

%Figure pptfsw_tinv
\begin{figure}[h]
        \begin{center}
                        \includegraphics[scale=0.5]{\figurepath pptfsw_tinv_dam.pdf}
		\caption{As in  Fig. \ref{pptflw_tinv_dam}, but for the SW band.
		\label{pptfsw_tinv_dam}                
		}
        \end{center}
\end{figure}

 %Figure pptfcs_tinv
\begin{figure}[h]
        \begin{center}
                        \includegraphics[scale=0.5]{\figurepath pptfc.pdf}
                \caption{\textbf{Left}: Clear-sky LW flux divergence  $-\ppt \FLWcs$ \textbf{Center:} Clear-sky SW flux divergence  $-\ppt \FSWcs$  \textbf{Right:} Clear-sky and all-sky  net flux divergence for the $\Ts=300$ K simulations, all plotted as in Fig. 2. The left and center panels are almost identical to the right panels of Figs. \ref{pptflw_tinv_dam} and \ref{pptfsw_tinv_dam}, and the right panel above shows directly the small difference between the all-sky and clear-sky flux divergences for the \Ts=300 K simulation. 
                \label{pptfc}
                }
        \end{center}
\end{figure}

%=====%
% RFM %
%=====%

%Figure tauk_tinv
\begin{figure}[h]
        \begin{center}
                        \includegraphics[scale=0.7]{\figurepath tauk_tinv.pdf}
                \caption{Optical depth profiles $\tauk(T)$, obtained by feeding thermodynamic profiles from the RCE simulations  into the RFM line-by-line radiative transfer code. Profiles are shown for water vapor only at three different wavelengths corresponding to surface optical depths of 0.01, 1, and 100 in the \Ts=280 K simulation. A reasonable degree of \Ts-invariance is seen at each wavelength: over the 30 K range of \Ts,  the temperature at which these lines reach $\tauk=1$ for example (where cooling-to-space is maximized) varies by at most 8 K, or a little over 25\% of the \Ts\ range.
                \label{tauk_tinv}
                }
        \end{center}
\end{figure}


%=====%
% GCM %
%=====%

%Figure fnetvar_ipsl
\begin{figure}[h]
        \begin{center}
                        \includegraphics[scale=0.7]{\figurepath fnetvar_ipsl.pdf}
                \caption{Variance of $-\ppt\Fnet$ within \Ts\ bins for the IPSL model. A fairly sharp pickup in the lower atmosphere is evident, which is then used to diagnose \Text\ (see SI text \ref{sec_pptfvar}), plotted in black dots.. 
                \label{fnetvar_ipsl}
                }
        \end{center}
\end{figure}

%Figure fnet_all_280
\begin{figure}[h]
        \begin{center}
                        \includegraphics[scale=0.6]{\figurepath fnet_all_280.pdf}
                \caption{As in Fig. 6 of the main text but for the $\Ts=280$ K bins.
                                \label{fnet_all_280}
                }
        \end{center}
\end{figure}


%Figure fnetcs_270_all
\begin{figure}[h]
        \begin{center}
                        \includegraphics[scale=0.6]{\figurepath fnetcs_270_all.pdf}
                \caption{All-sky and clear-sky $-\ppt\Fnet$ profiles for the AMIP case for all models for the $\Ts=270$ K bin. The majority of models show a significant near-surface CRE.
                                \label{fnetcs_270_all}
                }
        \end{center}
\end{figure}

%Figure fnetcs_290_all
\begin{figure}[h]
        \begin{center}
                        \includegraphics[scale=0.6]{\figurepath fnetcs_290_all.pdf}
                \caption{As in Fig. \ref{fnetcs_270_all},  but for the $\Ts= 290$ K bin. The near-surface CRE is much less significant across models than for the $\Ts= 270$ K bin.
                                \label{fnetcs_290_all}
                }
        \end{center}
\end{figure}

%Figure rh_all
\begin{figure}[h]
        \begin{center}
                        \includegraphics[scale=0.6]{\figurepath rh_all.pdf}
                \caption{RH profiles for our CFMIP models for the $\Ts=290$ (AMIP) and  $\Ts= 294$ K (AMIP4K) bins, computed just as for radiative fluxes. Like the $-\ppt\Fnet$ profiles, the RH profiles show \Ts-invariance aloft, but have lower-tropospheric features which shift downward (in temperature space) with warming.
                                \label{rh_all}
                }
        \end{center}
\end{figure}

%Figure fswlw_all
\begin{figure}[h!]
	\begin{center}
			\includegraphics[scale=0.6]{\figurepath fswlw_all.pdf}
		\caption{ Profiles of $-\ppt \FSW$ and $-\ppt \FLW$ for the \Ts=290 K (AMIP) and 294 K (AMIP4K) bins for all six CFMIP models. This is similar to Fig. 6 of the main text, but with the flux divergence decomposed into the LW and SW bands to show  that \Ts-invariance in the mid and upper troposphere holds across models holds for both the LW and SW separately, just as for the CRM. 
		\label{fswlw_all}
		}
	\end{center}
\end{figure}


%  %Figure fnet_HadGEM2-A
%  \begin{figure}[h]
%        \begin{center}
%                        \includegraphics[scale=0.4]{\figurepath fnet_HadGEM2-A.pdf}
%                \caption{As in previous figure, but for HadGEM2-A.
%                                \label{fnet_HadGEM2-A}
%                }
%        \end{center}
%  \end{figure}
%  
%  %Figure fnet_MIROC5
%  \begin{figure}[h]
%        \begin{center}
%                        \includegraphics[scale=0.4]{\figurepath fnet_MIROC5.pdf}
%                \caption{As in previous figure, but for MIROC5.
%                                \label{fnet_MIROC5}
%                }
%        \end{center}
%  \end{figure}
%  
%  %Figure fnet_MPI-ESM-LR
%  \begin{figure}[h]
%        \begin{center}
%                        \includegraphics[scale=0.4]{\figurepath fnet_MPI-ESM-LR.pdf}
%                \caption{As in previous figure, but for MPI-ESM-LR.
%                                \label{fnet_MPI-ESM-LR}
%                }
%        \end{center}
%  \end{figure}

		
\end{document}


